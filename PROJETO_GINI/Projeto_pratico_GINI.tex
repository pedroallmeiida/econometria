%% This document created by Scientific Word (R) Version 3.0



\documentclass[thmsa,11pt]{article}

\usepackage[utf8]{inputenc}% acentuação (se esse não funcionar use a linha abaixo)
%\usepackage[latin1]{inputenc}% acentuação (windows)
\usepackage[T1]{fontenc} % Use 8-bit encoding that has 256 glyphs
\usepackage{hyperref}
\usepackage[portuguese]{babel}
\usepackage{amsmath}
\usepackage{amsfonts}
\usepackage{amssymb}
\usepackage{graphicx}
\usepackage{comment}

\usepackage{tikz, xcolor}


\usepackage{algorithmic}
\usepackage{algorithm}
\usepackage{natbib}
\usepackage{subfig}
\usepackage{graphicx}
\usepackage{bm}
\usepackage{amsmath,amssymb}
\usepackage{longtable}
\usepackage{tcolorbox}
\usetikzlibrary{shadows}
\usepackage{colortbl} % Pacote necessário para \cellcolor
\usepackage{xcolor} % Adiciona suporte para cores

%TCIDATA{OutputFilter=latex2.dll}
%TCIDATA{TCIstyle=article/art4.lat,lart,article}
%TCIDATA{CSTFile=article.cst}
%TCIDATA{Created=Tue Jan 16 17:54:47 2001}
%TCIDATA{LastRevised=Sun Feb 16 19:37:46 2003}
%TCIDATA{<META NAME="GraphicsSave" CONTENT="32">}
\setlength{\textheight}{19.5cm}
\newtheorem{teorema}{Teorema}[subsection]
\newtheorem{obs}[teorema]{Observa\c{c}\~{a}o}
\newtheorem{defini}[teorema]{Defini\c{c}\~{a}o}
\newtheorem{prop}[teorema]{Proposi\c{c}\~{a}o}
\newtheorem{corolario}[teorema]{Corol\'{a}rio}
\newtheorem{lema}[teorema]{Lema}
\newtheorem{exercicio}[teorema]{Questão}
\newenvironment{demons}{\noindent{\bf Demonstra\c{c}\~ao:} }{\hfill $\Box$ \newline}
\newenvironment{demonsembox}{\noindent{\bf Demonstra\c{c}\~ao:} }{}
\newenvironment{exemplos}{\noindent {\bf Exemplos:} }{\hfill $\Box$ \newline}
\newenvironment{exemplo}{\vspace{12pt} \noindent{\bf Exemplo:} }{\hfill $\Box$ \newline}
\newenvironment{exemplosembox}{\vspace{12pt} \noindent {\bf Exemplos:} }{}
\newenvironment{demdoteo}{\noindent {\bf Demonstra\c{c}\~{a}o do Teorema}}{\hfill $\Box$ \newline}
\newenvironment{demdolem}{\noindent {\bf Demonstra\c{c}\~{a}o do Lema}}{\hfill $\Box$ \newline}
\newcommand{\adj}{\mathop{\rm ad}\nolimits}
\newcommand{\diag}{\mathop{\rm diag}\nolimits}
\newcommand{\esseo}{\mathop{\rm SO}\nolimits}
\newcommand{\gera}{\mathop{\rm ger}\nolimits}
\newcommand{\ident}{\mathop{\rm id}\nolimits}
\newcommand{\imag}{\mathop{\rm im}\nolimits}
\newcommand{\partre}{\mathop{\rm Re}\nolimits}
\newcommand{\partim}{\mathop{\rm Im}\nolimits}
\newcommand{\trac}{\mathop{\rm tr}\nolimits}
\hyphenation{con-si-de-ran-do-se ca-sa}


\begin{document}


\author{Professor: Pedro M.A. Junior}
\title{PROJETO PRÁTICO \\ Modelo de regressão beta aplicado ao índice de Gini}


\maketitle


Para a realização do projeto prático vamos utilizar os dados socioeconômicos no repositório \url{https://github.com/pedroallmeiida/econometria/PROJETO_GINI/dados_tratado/dados_final_projeto.csv}. 
De acordo com o banco de dados selecionado construa um modelo de regressão beta para avaliar o comportamento da variável resposta chamada GINI. A seguir vamos apresentar uma descrição das variáveis que vamos utilizar: 

\begin{table}[htb!]
	\tiny
	\caption{Descrição das variáveis}\label{tab:descricao}	
	\centering
	\begin{tabular}{|c|l|}
		\hline
		\verb|GINI| & Índice de Gini \\
		\verb|POP| & Quantidade de habitantes \\ 
		\verb|COD_MUNICIPIO| & Código do município \\
		\verb|NM_MUNICIPIO| & Nome do município \\
		\verb|UF| & Unidade de Federação do município \\
		\verb|AGUA_ENCANADA_PRINCIPAL| & Taxa de domicílios que possui ligação a rede geral de água  \\
		& \, e a utiliza como forma principal \\
		\verb|NAO_POSSUI_AGUA_ENCANADA| & Taxa de domicílios que não possui ligação da água com a rede geral \\
		\verb|REDE_GERAL_REDE_PLUVIAL_OU_FOSSA_LIGADA_A_REDE| & Taxa de domicílios com coleta adequada de esgoto \\
		\verb|NAO_TINHAM_BANHEIRO_NEM_SANITARIO| & Taxa de domicílios que não possuem banheiro nem sanitário \\
		\verb|PERC_COLETA_LIXO| & Percentual de domicílios com coleta de lixo\\
		\verb|PERC_ACESSO_INTERNET| & Percentual de domicílios com acesso a internet \\
		\verb|PERC_ALFA_15MAIS| & Percentual de pessoas alfabetizadas com 15 anos ou mais\\
		\verb|QTD_FORM_SUP_EDUC| & Quantidade de pessoas com curso superior em educação \\
		\verb|QTD_FORM_SUP| & Quantidade de pessoas com curso superior \\
		\verb|TX_CRESC_GEO| & Taxa de crescimento geométrico\\
		\verb|DENSIDADE_DEMO| & Densidade demográfica \\
		\verb|INDICE_ENVELHECIMENTO| & Índice de envelhecimento \\
		\verb|POP_PRETA| & População de pessoas pretas\\
		\verb|POP_PARDA| & População de pessoas pardas \\
		\verb|PIB_BRUTO| & Produto interno bruto\\
		\verb|PARTICIPACAO_PIB_UF| & Participação do PIB na Unidade de Federação\\
		\verb|PARTICIPACAO_PIB_MESOREGIAO| & Participação do PIB na Mesoregião \\
		\verb|TX_PESSOAS_PRETA_OU_PARDA| & Taxa de pessoas pretas ou pardas \\
		\verb|PIB_POR_HABITANTE| & PIB per capita\\
		\verb|TX_HOMICIDIO| & Taxa de homicídios\\
		\verb|TX_SUICIDIO| & Taxa de suicídio\\
		\verb|EXPORTACAO_PERCAPITA| & Exportação per capita\\
		\verb|POUPANCA_PERCAPITA| & Poupança per capita\\
		\verb|ARRECADACAO_TOTAL_IMPOSTOS_PERCAPITA| & Arrecadação total impostos per capita\\
		\verb|DESP_CIENCIA_TEC_PERCAPITA| & Despesas com ciência e tecnologia per capita\\
		\verb|VALOR_TOTAL_BF_PERCAPITA| & Valor total Gasto com Bolsa Família per capita\\
		\verb|DESP_LEGISLATIVO_PERCAPITA| & Despesa com o Legislativo per capita \\
		\verb|RECEITA_IPTU_PERCAPITA| & Receita IPTU per capita\\
		\verb|COTA_PARTE_FPM_PERCAPITA| & Cota Parte Fundo de Participação dos Municípios per capita \\
		\verb|DESP_SAUDE_SANEAMENTO_PERCAPITA| & Despesa com saúde saneamento per capita\\
		\verb|OPERACOES_CREDITO_PERCAPITA| & Operações de crédito per capita\\
		\verb|INDICADOR_REND_P_FUND| & Indicador de rendimento P para o ensino fundamental\\
		\verb|SAEB_MEDIA_FUND| & Nota média da SAEB para o ensino fundamental\\
		\verb|IDEB_2021_FUND| & Nota do IDEB para o ensino fundamental\\
		\verb|INDICADOR_REND_P_ENS_MEDIO| & Indicador de rendimento P para o ensino médio\\
		\verb|SAEB_MEDIA_ENS_MEDIO| & Nota média da SAEB para o ensino médio\\
		\verb|IDEB_2021_ENS_MEDIO| & Nota do IDEB para o ensino médio\\
		\hline
	\end{tabular}
\end{table}


\bigskip

\textbf{O projeto deve cumprir as seguintes etapas: }

\begin{enumerate}
	\item Objetivo e introdução;
	\item Estudo descritivo das variáveis utilizadas;
	\item Indique quais variáveis têm associação significativa com a variável resposta (Gini);
	\item Apresente o modelo final e suas métricas finais;
	\item Interprete os resultados obtidos pelo modelo;
	\item Conclua destacando seus principais resultados;
	\item Organize um repositório no seu github com todos os arquivos necessário para rodar o projeto (scripts no R, notebooks, slides, relatórios, etc ...)
\end{enumerate}



\end{document}